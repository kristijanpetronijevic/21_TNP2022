\documentclass[a4paper]{article}
\usepackage[utf8]{inputenc}
\usepackage{underscore}
\usepackage[english,serbian]{babel}
\usepackage{graphicx}
\usepackage{color}
\usepackage{url}
\usepackage{float}

\usepackage[unicode]{hyperref}
\hypersetup{colorlinks, citecolor=green, filecolor=green, linkcolor=blue, urlcolor=blue}

\begin{document}



\title{Pametni (elektricni) automobili u 2022. godini\vspace{3ex}\\
\small{Seminarski rad u okviru kursa\\
Tehničko i naučno pisanje\\
Matematički fakultet}\vspace{3ex}}
\author{\\\\\\Nera Zejak\\Eleonora Jovanovic Kisseleva\\Nemanja Potic\\Kristijan Petronijevic \\ \\nerazejak1130@gmail.com\\eleonorajovanovic1@gmail.com\\x.nemanjapotic.x@gmail.com\\ petronijevick3@gmail.com\\}
\date{13.~novembar 2022.}\
\maketitle

\maketitle



\abstract{
    U narednom tekstu cemo se ukratko upoznati sa \textbf{Pametnim (elektricnim) automobilima.} Pocev od toga sta su pametni automobili, objasnićemo šta su osnove bioinformatike, kao i mnoge zanimljive činjenice ove jedinstvene interdisciplinarne oblasti. Razgovaramo o glavnim principima koji podupiru bioinformatičke analize, razmatramo tipove bioloških informacija i baza podataka koje se obično koriste.}

\newpage

\tableofcontents
\newpage

\section{Uvod}
\abstract{
    Proba nekog teksta ovo ovde je \textbf{boldovano} ok }
\label{sec:uvod}
\end{document}
