\documentclass[a4paper]{article}

\usepackage{color}
\usepackage{url}
\usepackage[T2A]{fontenc} % enable Cyrillic fonts
\usepackage[utf8]{inputenc} % make weird characters work
\usepackage{graphicx}

\usepackage[english,serbian]{babel}
%\usepackage[english,serbianc]{babel} %ukljuciti babel sa ovim opcijama, umesto gornjim, ukoliko se koristi cirilica

\usepackage[unicode]{hyperref}
\hypersetup{colorlinks,citecolor=green,filecolor=green,linkcolor=blue,urlcolor=blue}

%\newtheorem{primer}{Пример}[section] %ćirilični primer
\newtheorem{primer}{Primer}[section]

\begin{document}

\title{Pametni (elektricni) automobili u 2022. godini\vspace{3ex}\\ \small{Seminarski rad u okviru kursa\\Tehničko i naučno pisanje\\ Matematički fakultet}}

\author{\\\\\\Nera Zejak\\Eleonora Jovanovic Kisseleva\\Nemanja Potic\\Kristijan Petronijevic \\ \\nerazejak1130@gmail.com\\eleonorajovanovic1@gmail.com\\x.nemanjapotic.x@gmail.com\\ petronijevick3@gmail.com\\}
\date{1.~novembar 2022.}
\maketitle

\abstract{
 U narednom tekstu cemo se ukratko upoznati sa \textbf{Pametnim (elektricnim) automobilima.} Pocev od toga sta su pametni automobili, objasnićemo šta su osnove bioinformatike, kao i mnoge zanimljive činjenice ove jedinstvene interdisciplinarne oblasti. Razgovaramo o glavnim principima koji podupiru bioinformatičke analize, razmatramo tipove bioloških informacija i baza podataka koje se obično koriste.}


\tableofcontents

\newpage

\section{Uvod}
   Električni automobil je automobil koji se pokreće elektromotorom, koristeći \textbf{električnu energiju} pohranjenu u akumulatoru, ili drugim uređajima za pohranu energije. Električni automobili su bili popularni krajem 19. i početkom 20. veka. 
        


\begin{figure}[h]
        \centering
        \includegraphics[width=\linewidth]{tesla.jpeg}
        \caption{Tesla model S}
        \label{fig:my_label}
        \end{figure}
        Unapređenja motora sa unutarnjim sagorevanjem i masovna proizvodnja jeftinijeg vozila na benzin doveli do smanjenja korištenja vozila na električni pogon.
         Energetske krize 1970-ih i 80-ih dovele su do kratkotrajnog zanimanja za \textbf{električne automobile}, te se sredinom 2000. obnovio interes za proizvodnjom električnih automobila, uglavnom zbog zabrinutosti oko ubrzanog povećanja cene nafte i potrebe za smanjenjem emisije gasova staklene bašte.
         
\label{sec:uvod}
\newpage

\section{Engleski termini i citiranje}	
\label{sec:termini_i_citiranje}

Na svakom mestu u tekstu naglasiti odakle tačno potiču informacije. Uz sve novouvedene termine u zagradi naglasiti od koje engleske reči termin potiče. 

Naredni primeri ilustruju način uvođenja enlegskih termina kao i citiranje.

\begin{primer}
Problem zaustavljanja (eng.~{\em halting problem}) je neodlučiv \cite{haltingproblem}.
\end{primer}

\begin{primer}
Za prevođenje programa napisanih u programskom jeziku C može se koristiti GCC kompajler \cite{gcc}.
\end{primer}

\begin{primer}
 Da bi se ispitivala ispravost softvera, najpre je potrebno precizno definisati njegovo ponašanje \cite{laski2009software}. 
\end{primer}

Ukoliko za unos referenci koriste datoteku {\em seminarski.bib},  prevođenje u pdf format u Linux okruženju može se uraditi na sledeći način:
\begin{verbatim}
pdflatex TemaImePrezime.tex 
bibtex TemaImePrezime.aux 
pdflatex TemaImePrezime.tex 
pdflatex TemaImePrezime.tex 
\end{verbatim}
Prvo latexovanje je neophodno da bi se generisao {\em .aux} fajl. {\em bibtex} proizvodi odgovarajući {\em .bbl} fajl koji se koristi za generisanje literature. 
Potrebna su dva prolaza (dva puta pdflatex) da bi se reference ubacile u tekst (tj da ne bi ostali znakovi pitanja umesto referenci). Dodavanjem novih referenci potrebno je ponoviti ceo postupak.  


Broj naslova i podnaslova je proizvoljan. Neophodni su samo Uvod i Zaključak. Na poglavlja unutar teksta referisati se po potrebi. 
\begin{primer}
U odeljku \ref{sec:naslov1} precizirani su osnovni pojmovi, dok su zaključci dati u odeljku \ref{sec:zakljucak}.
\end{primer}




\section{Slike i tabele}
\label{slike_i_tabele}

Slike i tabele treba da budu u svom okruženju, sa odgovarajućim naslovima, obeležene labelom da koje omogućava referenciranje. 

\begin{primer} Ovako se ubacuje slika. Obratiti pažnju da je dodato i 
\begin{verbatim}
\usepackage{graphicx}
\end{verbatim}

\begin{figure}[h!]
\begin{center}
\includegraphics[scale=0.75]{pande.jpg}
\end{center}
\caption{Pande}
\label{fig:pande}
\end{figure}

Na svaku sliku neophodno je referisati se negde u tekstu. Na primer, na slici \ref{fig:pande} prikazane su pande. 
\end{primer}

\begin{primer} I tabele treba da budu u svom okruženju, i na njih je neophodno referisati se u tekstu. Na primer, u tabeli \ref{tab:tabela1} su prikazana različita poravnanja u tabelama.

\begin{table}[h!]
\begin{center}
\caption{Procenat kupljenih automobila sa elektricnim pogonom za 2022 godinu.}
\begin{tabular}{|c|c|c|} \hline
Norveška& Island& Švedska\\ \hline
58\% &18\%&11\%\\ \hline
\end{tabular}
\label{tab:tabela1}
\end{center}
\end{table}

\end{primer}





\section{Istorija}
\label{sec:naslov1}


Električna vozila su u upotrebi već skoro 2 veka i ona se smatraju sledećim korakom ka održivijem i više ekološki prihvatljivijom urbanom transportu. Njihova popularnost i upotreba raste od početka 21. veka, kada je interesovanje za njih poraslo zbog uticaja na životnu sredinu od strane emisija koja prave vozila koja rade na fosilnim gorivima. Potražnja za električnim automobilima će nastaviti da raste zbog toga što potrošači uvek traže način da smanje troškove, a cene električnih automobila vremenom opadaju i njihovom upotrebom nema potrebe za dodatnim troškovima na goriva.


\subsection{Rana istorija}
\label{subsec:podnaslov1}

Praktična električna vozila su se prvi put pojavila 90-ih godina 19. veka i ona su držala rekord u brzini vozila do oko 1900-te godine. Električna energija je bila među poželjnijim metodoma automobilskog pogona jer je pružala nivo lakoće rada i udobnosti koju automobili na benzin tog vremena nisu mogli da pruže.

Električna vozila su se koristila kao privatna motorna vozila do 20. veka, kada je zbog niskih maksimalnih brzina i visokih cena u poređenju sa vozilima sa SUS motorima (motorima sa unutrašnjim sagorevanjem), došlo do svetskog pada njihove upotrebe kao privatnih motornih vozila. Odlučujući momenat u ovom preokretu je bilo uvođenje elektropokretača tj. anlasera 1912. godine. koji je zamenio druge metode pokretanja SUS motora. Jedna od zamenjenih metoda je bila pokretanje kurblom tj. valjkastom šipkom od gvožđa koju je bilo potrebno okretati naglo i snažno sve dok se SUS motor ne bi upalio.

Električna vozila su se idalje koristila za javni prevoz i kao utovarna i teretna oprema, ali se tek u na početku 21. veka interesovanje za njihovu upotrebu kao privatna vozila ponovo povećalo.


\subsection{Moderni električni automobili}
\label{subsec:podnaslov2}


Na početku 21. veka je došlo do porasta zabrinutosti zbog štete po životnu sredinu urzokovanu emisijama koje prave vozila koja rade na fosilnim gorivima. CARB (California Air Reasources Board) tj. agencija za čist vazduh u vladi Kalifornije se već u ranim 90-im godinama 20. veka borila za upotrebu vozila koja ispuštaju manje štetnih gasova, sa krajnjim ciljem da se koriste vozila koja ih uopšte neće ispuštati, kao što su, na primer, električna.

Dobar deo zasluge za povećanje interesovanja za električne automobile je bilo prouzrokovano od strane 2 događaja.

Prvi događaj je bio pojava Toyota Prius-a, modela koji je počeo da se porizvodi u Japanu 1997. godine. On je postao prvo svetsko hibridno električno vozilo koje je imalo masivnu proizvodnju. U 2000. godini je počeo da se prodaje širom sveta i vrlo brzo je postao popularan među poznatim ličnostima, što je znatno podiglo profil modela. Za njegovu proizvodnju Toyota je koristila nikl-metal hidridnu bateriju, tehnologiju podržanu od strane odeljenja za istraživanje energije. Prius je postao najprodavaniji hibrid širom sveta u poslednjoj deceniji zahvaljujući porastu cena benzina i zabrinutosti zbog zagađenja ugljenikom.

Drugi događaj je bila vest u 2006. godini o tome da će kompanija Tesla Motors, danas poznatija kao Tesla, početi da proizvodi model luksuznih električnih sportskih automobila koji će moći da idu više od 320 km sa jednim punjenjem. Ona je već 2004. godine krenula  sa razvojem datog automobila, a 2008. godine ga je i dostavila klijentima. Ime modela je Tesla Rodster i on je prvi potpuno električni automobil legalan na autoputu i prvi proizvodni potpuno električni automobil koji putuje više od 320 km.

Kompanija Mitsubishi Motors je 2009. godine u Japanu počela da prodaje Mitsubishi i-MIEV, prvi električni automobil legalan na autoputu koji se serijski proizvodio. Ovaj model automobila je takođe prvi koji je prodat u više od 10.000 primeraka.

U 2008. godini su počele promene u proizvodnji električnih automobila zbog napredaka baterija, cilja da se smanje emisije gasova staklene bašte i da se poboljša kvalitet vazduha.

Tesla Model 3 je u martu 2020. godine prestigao Nissan Leaf i postao najprodavaniji električni automobil svih vremena, sa više od 500.000 prodatih primeraka. U junu 2021. godine je dostigao 1.000.000 globalno prodatih primeraka.


\subsection{Budućnost električnih automobila}
\label{subsec:podnaslov3}


Ne možemo tačno utvrditi šta sledi električhnim automobilima u budućnosti, ali trenutno oni imaju ogroman potencijal da učine tu budućnost ekološki održivijom. U transportnom sektori oni bi mogli da potpuno zamene fosilna goriva, da povećaju energetsku efikasnost i da smanje zagađenje.

U pitanje se dovode i potencijalni problemi koje bi korišćenje samo električnih vozila dovelo. Jedan od problema je pitanje dugoročne održivosti električnih vozila zbog rizika koji predstavlja nabavka kritičnih materijalnih resursa koji se koriste u baterijama električnih automobila. Eksploatacija nekih od ovih resursa je povezana sa značajnim uticajima na životnu sredinu, kao i sa društvenim i etičkim pitanjima. 



\section{MODELI AUTOMOBILA KOJI SU OBELEŽILI 2022. GODINU\vspace{2ex}}
\label{sec:MODELI AUTOMOBILA KOJI SU OBELEŽILI 2022. GODINU}

   Prošla godina je protekla kao i 2020. u borbi sa pandemijom, ali je svaka fabrika prolazila i kroz svoje probleme koji su se odnosili na nestašicu čipova. Drugu polovinu 2021. godine obeležio je i pad proizvodnje i prodaje automobila. Izostale su takođe i brojne premijere novih modela koje su odložene za ovu godinu. SUV vozila ostaju i dalje popularna i najtraženija, a očigledno je da ćemo viđati sve više električnih automobila sa sve većim dometom.\\\\

    Ono što je novina u autoindustriji je da će još jedan sistem postati deo obavezne opreme automobila. Sa tom uredbom se krenulo od jula ove godine. Evropska komisija usvojila je nacrt uredbe kojom se definiše obavezna ugradnja uređaja za “inteligentno prilagođavanje brzine” (Intelligent Speed Assistance – ISA) na svim novim vozilima i to od 6. jula 2022. godine. Sve nas naravno interesuje i kakve će boje biti aktuelne. Proteklih godina dominirale su siva i crna, a zatim je iznenada bela postala omiljena kod kupaca. Naročito je tražena kad su u pitanju SUV vozila. Što se tiče boja mnogi su dali titulu  trendsetera Hyundai modelima.
    \\\\
    Ovo su neki od modela sa kojima ćemo se upoznati, a koje su predstavili poznate ili neke novije auto kuće:\\
    \begin{itemize}
     \item ŠKODA ENYAQ IV 
     \item RENAULT AUSTRAL   
     \item MERCEDES-BENZ VISION EQXX 
     \item EVOLUTE I-PRO 
     \item DRAKO DRAGON\\ 
    \end{itemize}
    
\maketitle{\textbf{ŠKODA ENYAQ IV}}
\\

    Škoda je predstavila svoj novi ENYAQ COUPÉ iV 31. januara 2022. godine. To je električni top model češkog proizvođača automobila. S koeficijentom otpora od 0,234, ENYAQ COUPÉ iV je predvodnik u svojoj klasi, što ga čini posebno efikasnim u vožnji. \\
    Poput modela ENYAQ iV, novi ŠKODA ENYAQ COUPÉ iV takođe se temelji na modularnoj platformi (MEB) Volkswagen grupe. S koeficijentom otpora od cd 0,234, elegantni coupé postaće predvodnika u svom segmentu i poboljšaće već odličnu vrednost modela ŠKODA ENYAQ iV, delom zahvaljujući novom obliku karoserije.\\ 

\begin{figure}[h]
        \centering
        \includegraphics[width=\linewidth]{ENYAQ COUPÉ iV.png}
        \caption{ENYAQ COUPÉ iV}
        \label{fig:my_label1}
        \end{figure}

\newpage
\maketitle{\textbf{RENAULT AUSTRAL}}
\\

    Renault za ovu godinu imao cilj da osvoji C-segment sa novim SUV-om pod nazivom Austral. Zadržane su temeljne karakteristike koje SUV-ove čine privlačnima. Ipak, novi dizajnerski pristup i novi oblici predstavljaju pomak od tradicionalnih i statičnih vodoravnih linija, postavljenih u ravni sa tlom. Austral se ističe oštrim i dinamičnim linijama, naglašenim bočnim linijama, ali i novim oblikom svjetala. Dva velika zadnja svetla u obliku slova C stapaju se sa središnjim logotipom.\\  

\begin{figure}[h]
        \centering
        \includegraphics[width=\linewidth]{Austral.jpg}
        \caption{RENAULT AUSTRAL}
        \label{fig:my_label2}
        \end{figure}

\newpage

\maketitle{\textbf{MERCEDES-BENZ VISION EQXX}}
\\

    Ovaj model se smatra za  “najefikasniji ikada izrađen Mercedes”. To nije prvi konstruisan novi masovno proizvedeni električni Mercedes EQS ili pak EQE, već potpuno nov automobil. On je dakle osmišljen na novim osnovama od samih početaka razvoja. Novitet predstavlja neku vrstu manifesta za diviziju Mercedes-EQ nemačke grupacije Daimler koja se bavi elektromobilima i kao takav daje uvid u smer razvoja kompanijskih baterijskih modela.\\ 
 

\begin{figure}[h]
        \centering
        \includegraphics[width=\linewidth]{Vision.png}
        \caption{MERCEDES-BENZ VISION EQXX}
        \label{fig:my_label3}
        \end{figure}


\newpage

\maketitle{\textbf{EVOLUTE I-PRO}}
\\

    U Rusiji je počela prodaja prvog električnog automobila proizvedenog u toj zemlji, saopštio je proizvođač MotorInvest, javlja Anadolija. Lansiranje prvih ruskih proizvedenih električnih vozila pod brendom Evolute održano je krajem septembra u zapadnoj regiji Lipeck. I-Pro limuzina je bila prvi model koji iz proizvodne trake pušten u prodaju.\\
    Evolute i-Pro limuzina je opremljena vučnom baterijom kapaciteta 53 kilovata (kW) i ostvaruje domet do 420 kilometara. Električni motor ima 150 konjskih snaga.
    Početna cena automobila je tri miliona rubalja (oko 50.000 dolara). Prodajna mreža MotorInvesta pokriva devet regiona Rusije: Moskvu, Sankt Peterburg, Nižnji Novgorod, Kazanj, Voronjež, Lipeck, Krasnodar, Rostov na Donu i Soči.\\
    Do kraja godine MotorInvest planira u prodaju pustiti i modele iJoy crossover te iVan minivan. Modeli će koštati 3,49 miliona rubalja, a 2023. će se pojaviti i-Jet cross-coupe. Ukupno, do kraja godine, kompanija namjerava proizvesti 2.000 električnih vozila.\\ 

\newpage

\maketitle{\textbf{DRAKO DRAGON}}
\\

    Kompanija Drako Motors je nedavno održala premijeru svog novog modela, a to je električni hyper-SUV Dragon. 
    Dragon SUV sa gullwing vratima je baziran na kompanijinoj arhitekturi Drako DriveOS i ima Drako DriveOS Quad Motor pogonski sistem. U pitanju je postavka sa četiri električna motora i ukupno 2000KS. Prema najavama, to će omogućiti ubrzanje od 0 60 milja na sat (96 km/h) za 1,9 sekundi.
    Najveća brzina ovog 2254 kg teškog vozila iznosi više od 200 milja na sat (322 km/h), dok četvrt milje (402 metra) može da se pređe za 9,0 sekundi. Što se dometa tiče, sa jednim punjenjem Dragon može da pređe do 676 km. Dragon je dugačak 5054 mm, postavljen je na točkove od 23 inča, a u kabini nudi mesta za pet putnika.
    Drako Motors već prima rezervacije za prvu seriju od 99 vozila. Cene startuju od 290.000 dolara, ali isporuke neće početi pre 2026. godine. U planu je da se kasnije proizvodi 5.000 primeraka godišnje.
 

\begin{figure}[h]
        \centering
        \includegraphics[width=\linewidth]{DRAKO.jpg}
        \caption{DRAKO DRAGON}
        \label{fig:my_label5}
        \end{figure}




\section{Zaključak}
\label{sec:zakljucak}

Ovde pišem zaključak. 
Ovde pišem zaključak. 
Ovde pišem zaključak. 
Ovde pišem zaključak. 
Ovde pišem zaključak. 
Ovde pišem zaključak. 
Ovde pišem zaključak. 
Ovde pišem zaključak. 
Ovde pišem zaključak. 
Ovde pišem zaključak. 
Ovde pišem zaključak. 
Ovde pišem zaključak. 


\addcontentsline{toc}{section}{Literatura}
\appendix

\iffalse
\bibliography{seminarski} 
\bibliographystyle{plain}
\fi

\begin{thebibliography}{9}

\bibitem{laski2009software} J. Laski and W. Stanley. \emph{Software Verification and Analysis}. Springer- Verlag, London, 2009.

\bibitem{gcc} Free Software Foundation. GNU gcc, 2013. on-line at: http://gcc. gnu.org/.

\bibitem{haltingproblem} A. M. Turing. \emph{On Computable Numbers, with an application to the Entscheidungsproblem}. Proceedings of the London Mathematical Society, 2(42):230–265, 1936.


\end{thebibliography}





\end{document}
